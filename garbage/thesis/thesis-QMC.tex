% QMC in the VB Basis Chapter
%======================================================================
%======================================================================
\chapter{Quantum Monte Carlo in the Valence Bond Basis}

- VB QMC is a T=0, GS projection technique (redundant? Yes)\\
%======================================================================
%======================================================================
\section{The Valence Bond Basis}
%What is a VB and how can it be represented?\\
%Representing multiple VBs\\
%VB basis properties\\

The valence bond basis, like the $S^Z$ basis, can be used to represent spin states.

\subsection{The Valence Bond}
The valence bond, or singlet state, is a maximally entangled state between two spin 1/2 particles
which together have zero spin angular momentum.  A valence bond can be represented in the $S^Z$ basis, as a list of the two sites joined, or 
visually as a bond joining two sites.  {\color{red} (mention direction of singlet? i.e. Unlike the triplet states, the order of the particles involved will change the phase of the state.)}

\begin{equation}
   (a,b) = \frac{1}{\sqrt{2}}\big( \lvert \uparrow_a \downarrow_b \rangle - \lvert \downarrow_a \uparrow_b \rangle \big)
   \label{eqn_vb}
\end{equation}

%--------------------------------------------------------------------------------------------------------------------------
\section{Ground State Projection}
%--------------------------------------------------------------------------------------------------------------------------
\section{The Hamiltonian / Bond Operators}
%--------------------------------------------------------------------------------------------------------------------------
\section{Single Projector}
\subsection{Measurements}
%--------------------------------------------------------------------------------------------------------------------------
\section{Double Projector}
%--------------------------------------------------------------------------------------------------------------------------
\section{Loop Moves}
%--------------------------------------------------------------------------------------------------------------------------
