% QMC in the VB Basis Chapter
%======================================================================
%======================================================================
\chapter{Quantum Monte Carlo in the Valence Bond Basis}

{\color{red} - VB QMC is a T=0, GS projection technique (redundant? Yes)}
\\
%======================================================================
%======================================================================
\section{The Valence Bond Basis}
%What is a VB and how can it be represented?\\
%Representing multiple VBs\\
%VB basis properties\\

{\color{red} The valence bond basis, like the $S^Z$ basis, can be used to represent spin states.}

\subsection{The Spin 1/2 Singlet State}


Typically the spin states of spin 1/2 particles are represented in the $S^Z$ basis.  Applying the
$S^Z$ operator to its eigenstates yields either $+1/2$ or $-1/2$ corresponding to the spin up
and spin down eigenstates respectively.
\begin{equation}
   S^Z\lvert \uparrow \rangle = +\frac{1}{2} \lvert \uparrow \rangle
   \:\:\:    \:\:\:    \:\:\:    \:\:\:    \:\:\:    \:\:\: 
   S^Z\lvert \downarrow \rangle = -\frac{1}{2} \lvert \downarrow \rangle
   \label{SZ}
\end{equation}

A singlet state refers the any spin state of a particle or group of particles with vanishing total spin angular momentum.
For two spin 1/2 particles there is exactly one singlet state, represented in the $S^Z$ basis as

\begin{equation}
  \frac{1}{\sqrt{2}}\left( \lvert \uparrow_1 \downarrow_2 \rangle - \lvert \downarrow_1 \uparrow_2 \rangle \right) .
   \label{singlet}
\end{equation}

  {\color{red} show that total spin is 0?
  
  (mention direction of singlet? i.e. Unlike the triplet states, the order of the particles involved will change the phase of the state.)

\begin{equation}
   \lvert(a,b)\rangle = \frac{1}{\sqrt{2}}\big( \lvert \uparrow_a \downarrow_b \rangle - \lvert \downarrow_a \uparrow_b \rangle \big) = ...
   \label{eqn_vb}
\end{equation}
}

{\color{red} y'know, make this whole thing better.}
A bond between two atoms created by sharing valence electrons in the outer orbitals is called a valence bond, or a covalent bond \cite{Slater1931,Pauling1933}. 
Since electrons are fermionic (have half integer spin) their total
wave function must be antisymmetric (an exchange of the identical particles in the wave function
gives a factor of -1, i.e. $\Psi(1,2) = -\Psi(2,1)$).
The total wave function is a product of the spatial and spin wave functions, so one of those wave functions must be antisymmetric and the other symmetric (an exchange of particles yields the same wave function, i.e. $\Psi(1,2) = \Psi(2,1)$).
As part of the same valence bond, two electrons will have a symmetric spatial wave function {\color{red} true?}, so their spin state must be antisymmetric.  For two spin 1/2 particles the only antisymmetric spin state is the singlet state.  
Hence, in the case of two spin 1/2 particles, a valence bond is equivalent to the spin 1/2 singlet state from equation (\ref{singlet}).

{\color{red} anything about it being maximally entangled?}

{\color{red} add in a picture of a valence bond... not sure how to draw that yet.

 Maybe add something about the VB spin or what it's an eigenstate of... total 
$S^Z$ and $S^2$

REFERENCES!!!!!!!!!!!!!!!
}

\subsection{Basis Properties}

- VB basis can be used to represent any singlet state\\
- spans the spin 0 sector (does it?  for all sizes? totally not for 2 sites. or 4?)\\
- only for an even number of sites\\
- singlets have a lower energy that the classical N\'eel state\\
- mention sign/directionality of VBs\\
- have to go from sublattice A to B (maybe put this in a different section?)\\
{\color{red} Fazekas and Anderson said it's a good choice for the heis s=1/2 antiferro GS
because it has total spin zero, and lower energy than the N\'eel state.  check if this is true.}

{\color{red} Thing from Anderson 1973 about energy per NN VB vs energy per 2 antiparallel sites,
but with math and stuff this time.}

\subsection{One more section, but on what?}


%--------------------------------------------------------------------------------------------------------------------------
\section{Ground State Projection}
%--------------------------------------------------------------------------------------------------------------------------
\section{The Hamiltonian / Bond Operators}
%--------------------------------------------------------------------------------------------------------------------------
\section{Single Projector}
\subsection{Measurements}
%--------------------------------------------------------------------------------------------------------------------------
\section{Double Projector}
%--------------------------------------------------------------------------------------------------------------------------
\section{Loop Moves}
%--------------------------------------------------------------------------------------------------------------------------
