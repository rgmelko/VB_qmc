% QMC in the VB Basis Chapter
%======================================================================
%======================================================================
\chapter{Quantum Monte Carlo in the Valence Bond Basis}

- VB QMC is a T=0, GS projection technique (redundant? Yes)\\
%======================================================================
%======================================================================
\section{The Valence Bond Basis}
%What is a VB and how can it be represented?\\
%Representing multiple VBs\\
%VB basis properties\\

{\color{red} The valence bond basis, like the $S^Z$ basis, can be used to represent spin states.}

\subsection{The Valence Bond}


Typically the spin states of spin 1/2 particles are represented in the $S^Z$ basis.  Applying the
$S^Z$ operator to its eigenstates yields either $+1/2$ or $-1/2$ corresponding to the spin up
and spin down eigenstates respectively.
\begin{equation}
   S^Z\lvert \uparrow \rangle = +\frac{1}{2} \lvert \uparrow \rangle
   \:\:\:    \:\:\:    \:\:\:    \:\:\:    \:\:\:    \:\:\: 
   S^Z\lvert \downarrow \rangle = -\frac{1}{2} \lvert \downarrow \rangle
   \label{SZ}
\end{equation}
{\color{red} something else?}

The valence bond, or singlet state, is a maximally entangled state between two spin 1/2 particles
which together have zero spin angular momentum.  A valence bond can be represented in the $S^Z$ basis, as a list of the two sites joined, or 
visually as a bond joining two sites.  {\color{red} (mention direction of singlet? i.e. Unlike the triplet states, the order of the particles involved will change the phase of the state.)}

\begin{equation}
   \lvert(a,b)\rangle = \frac{1}{\sqrt{2}}\big( \lvert \uparrow_a \downarrow_b \rangle - \lvert \downarrow_a \uparrow_b \rangle \big) = 
   \label{eqn_vb}
\end{equation}
{\color{red} add in a picture of a valence bond... not sure how to draw that yet.}  

{\color{red} Maybe add something about the VB spin or what it's an eigenstate of... total 
$S^Z$ and $S^2$}

{\color{red} Fazekas and Anderson said it's a good choice for the heis s=1/2 antiferro GS
because it has total spin zero, and lower energy than the N\'eel state.  check if this is true.}

{\color{red} Thing from Anderson 1973 about energy per NN VB vs energy per 2 antiparallel sites,
but with math and stuff this time.}


%--------------------------------------------------------------------------------------------------------------------------
\section{Ground State Projection}
%--------------------------------------------------------------------------------------------------------------------------
\section{The Hamiltonian / Bond Operators}
%--------------------------------------------------------------------------------------------------------------------------
\section{Single Projector}
\subsection{Measurements}
%--------------------------------------------------------------------------------------------------------------------------
\section{Double Projector}
%--------------------------------------------------------------------------------------------------------------------------
\section{Loop Moves}
%--------------------------------------------------------------------------------------------------------------------------
