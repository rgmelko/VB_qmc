% QMC in the VB Basis Chapter
%======================================================================
%======================================================================
\chapter{Quantum Monte Carlo in the Valence Bond Basis}

{\color{red} - VB QMC is a T=0, GS projection technique (redundant? Yes)}
\\
%======================================================================
%======================================================================
\section{The Valence Bond Basis}
%What is a VB and how can it be represented?\\
%Representing multiple VBs\\
%VB basis properties\\

{\color{red} The valence bond basis, like the $S^Z$ basis, can be used to represent spin states.}

\subsection{The Spin 1/2 Singlet State}


Typically the spin states of spin 1/2 particles are represented in the $S^Z$ basis.  Applying the
$S^Z$ operator to its eigenstates yields either $+1/2$ or $-1/2$ corresponding to the spin up
and spin down eigenstates respectively.
\begin{equation}
   S^z\lvert \uparrow \rangle = +\frac{1}{2} \lvert \uparrow \rangle
   \:\:\:    \:\:\:    \:\:\:    \:\:\:    \:\:\:    \:\:\: 
   S^z\lvert \downarrow \rangle = -\frac{1}{2} \lvert \downarrow \rangle
   \label{SZ}
\end{equation}

A singlet state refers to a spin state of a particle or group of particles with vanishing total spin angular momentum.
%other states are not eigenstates of S^2
For two spin 1/2 particles there is exactly one singlet state, represented in the $S^Z$ basis as

\begin{equation}
  \frac{1}{\sqrt{2}}\left( \lvert \uparrow_1 \downarrow_2 \rangle - \lvert \downarrow_1 \uparrow_2 \rangle \right) .
   \label{singlet}
\end{equation}

  {\color{red} show that total spin is 0? (in scribbly notebook, Sept 14th)
  
  (mention direction of singlet? i.e. Unlike the triplet states, the order of the particles involved will change the phase of the state.  Will have to do this later anyway.)

\begin{equation}
   \lvert(a,b)\rangle = \frac{1}{\sqrt{2}}\big( \lvert \uparrow_a \downarrow_b \rangle - \lvert \downarrow_a \uparrow_b \rangle \big) = ...
   \label{eqn_vb}
\end{equation}
}

{\color{red} y'know, make this whole thing better.}
A bond between two atoms created by sharing valence electrons in the outer orbitals is called a valence bond, or a covalent bond \cite{Slater1931,Pauling1933}.
Since electrons are fermionic (have half integer spin) their total
wave function must be antisymmetric (an exchange of the identical particles in the wave function
gives a factor of -1, i.e. $\Psi(1,2) = -\Psi(2,1)$).
The total wave function is a product of the spatial and spin wave functions, so one of those wave functions must be antisymmetric and the other symmetric (an exchange of particles yields the same wave function, i.e. $\Psi(1,2) = \Psi(2,1)$).
As part of the same valence bond, two electrons will have a symmetric spatial wave function {\color{red} true?}, so their spin state must be antisymmetric.  For two spin 1/2 particles the only antisymmetric spin state is the singlet state.  
Hence, in the case of two spin 1/2 particles, a valence bond is equivalent to the spin 1/2 singlet state from equation (\ref{singlet}).

{\color{red} anything about it being maximally entangled?}

{\color{red} add in a picture of a valence bond... not sure how to draw that yet.

 Maybe add something about the VB spin or what it's an eigenstate of... total 
$S^Z$ and $S^2$

REFERENCES!!!!!!!!!!!!!!!
}

\subsection{Basis Properties}

A collection of sites on a lattice can be paired into a valence bonds such that each site
belongs to exactly one bond.  We call this a valence bond covering (depicted in {\color{red}
Figure Whatever}).  
{\color{red} Totally put in a VB covering figure}
A basis of valence bond coverings can be used to represent an arbitrary singlet state
for an even number of spins.
The number of possible singlet states for a given number of spin 1/2 sites can be enumerated
using the rule for the addition of angular momentum for each added spin,
\begin{equation}
S\otimes \tfrac{1}{2}  = \left(S-\tfrac{1}{2}\right)\oplus\left(S+\tfrac{1}{2}\right),
\end{equation}
where S is the $(2S+1)$-degenerate state of spin $S$ \cite{Beach2006}.
\begin{eqnarray}
\tfrac{1}{2} \otimes \tfrac{1}{2}  &=& 0 \oplus 1\nonumber \\ 
\tfrac{1}{2} \otimes \tfrac{1}{2}  \otimes \tfrac{1}{2} &=& 
	\tfrac{1}{2} \oplus  \tfrac{1}{2} \oplus  \tfrac{3}{2} \nonumber \\
\tfrac{1}{2} \otimes \tfrac{1}{2}  \otimes \tfrac{1}{2} \otimes \tfrac{1}{2} &=& 
	0 \oplus 0 \oplus 1 \oplus 1 \oplus 1 \oplus 2 \nonumber \\
\tfrac{1}{2} \otimes \tfrac{1}{2}  \otimes \tfrac{1}{2}  \otimes \tfrac{1}{2}  \otimes \tfrac{1}{2}&=& 
	\tfrac{1}{2} \oplus \tfrac{1}{2} \oplus \tfrac{1}{2} \oplus \tfrac{1}{2} \oplus \tfrac{1}{2} \oplus  
	\tfrac{3}{2} \oplus \tfrac{3}{2} \oplus \tfrac{3}{2} \oplus \tfrac{3}{2} \oplus
	  \tfrac{5}{2} \nonumber \\
\tfrac{1}{2} \otimes \tfrac{1}{2}  \otimes \tfrac{1}{2} \otimes \tfrac{1}{2} \otimes \tfrac{1}{2}
\otimes \tfrac{1}{2} &=& 
	\underbrace{0 \oplus\cdots \oplus 0}_{5 \rm{\; times}} \oplus 
	\underbrace{1 \oplus \cdots \oplus 1}_{9 \rm{\; times}} \oplus 
	\underbrace{2 \oplus \cdots \oplus 2}_{5 \rm{\; times}} 
	\oplus 3 \nonumber
\end{eqnarray}
For and even number of spins, 2N, the number of singlet states is given by 
\begin{equation}
C_{\rm{sing}}^N = \frac{1}{N+1}\binom{2N}{N}\ = \frac{(2N)!}{N!(N+1)!},
\end{equation}  
and each of these singlet states is linearly independent of the others.
{\color{red} Can I figure out what the states actually look like?  Why is it 1/(n+1)(2n choose n)?}
In contrast, the number of possible valence bond states is given by
\begin{equation}
C_{\rm{VB}}^N =
	\frac{(2N)!}{2^NN!},
\end{equation}
since we choose sites at random ($2N!$ ways to choose them) pairing them, but the order 
in which each member of the bond is chosen does not matter (divide by 2 for every bond), 
nor does the order in which the $N$ bonds are chosen (divide by $N!$).
For $N>1$ there more valence bond coverings than singlet states, the excess increasing 
drastically with increasing $N$.  
The valence bond states are not orthogonal to each other, in fact every valence bond covering
has some overlap with every other covering.
Because of this overcompleteness we can eliminate some of the valence bond states and still
represent any singlet state.
It is convenient to define two sublattices, A and B, on a bipartite lattice, such that sites on 
sublattice A are neighbored only by sublattice B sites and vice versa. 
{\color{red} diagram of sublattices on a square lattice or something.}
We can choose the valence bond coverings containing only bonds going from sublattice A
to B.  (I say "from A to B" because valence bonds are, in fact, directional.  The ordering of the
sites matters, and if the order is switched we get the same state with a factor of -1.)
This restriction eliminates some, though not all, of the overcompleteness of the states.
The, now reduced, number of valence bond states is simply $C_{\rm{AB}}^N = N!$.
{\color{red} graph comparing the functions}
{\color{red} maybe show some of my awesome matrix calculations? $S^2$ for N=2 and N=3, and 
the lack of linear independence for N=3.}

{\color{red}
- maybe include that maple graph\\
- then actually go over the properties if they haven't already been mentioned.

- VB basis can be used to represent any singlet state\\
- spans the spin 0 sector\\
- only for an even number of sites\\
- singlets have a lower energy that the classical N\'eel state\\
- mention sign/directionality of VBs\\
- have to go from sublattice A to B (maybe put this in a different section?)\\
- show how we represent VB's and VB coverings
}

{\color{red} Fazekas and Anderson said it's a good choice for the heis s=1/2 antiferro GS
because it has total spin zero, and lower energy than the N\'eel state.  check if this is true.}

{\color{red} Thing from Anderson 1973 about energy per NN VB vs energy per 2 antiparallel sites,
but with math and stuff this time.}

\subsection{One more section, but on what?}


%--------------------------------------------------------------------------------------------------------------------------
\section{Ground State Projection}

Quantum Monte Carlo in the valence bond basis is a ground state projection technique, 
which means we start with a trial wave function apply high powers of the Hamiltonian until 
we are left with the zero temperature ground state of the system.  {\color{red} Is this off topic $-->$}
Though, being that it is a
Monte Carlo technique, we are not actually left with the wave function, but we sample terms in
the ground state wave function and we can then sample observable properties
 according to the statistics of the 
ground state wave function.\\
{\color{red} explain GS projection real good now.}

Though valence bond states are not necessarily represented uniquely, an arbitrary state can 
always be represented as a unique combination of the energy eigenstates of a Hamiltonian,
\begin{equation}
\lvert \psi \rangle = \sum_n c_n \lvert n \rangle,
\label{state}
\end{equation}
where $\lvert n \rangle$ is the $n^{\rm{th}}$ energy eigenstate of a Hamiltonian and the 
$c_n$\!'s are
the unique coefficients.

If we apply the Hamiltonian to the state in Eq.~(\ref{state}) we are left with
\begin{equation}
\mathcal{H}\lvert \psi \rangle = \sum_n c_n \mathcal{H} \lvert n \rangle =
 		\sum_n c_n E_n \lvert n \rangle = 
		c_0 E_0 \lvert 0 \rangle + c_1 E_1 \lvert 1 \rangle +
		c_2 E_2 \lvert 2 \rangle + \cdots,
\end{equation}
where $E_n$ is the $n^{\rm{th}}$ energy eigenvalue of the Hamiltonian, $\mathcal{H}$.
We can then take out a factor of $E_0$ to get
\begin{equation}
\mathcal{H}\lvert \psi \rangle =
		E_0 \left(c_0 \lvert 0 \rangle + c_1 \frac{E_1}{E_0} \lvert 1 \rangle +
		c_2\frac{ E_2}{E_0} \lvert 2 \rangle + \cdots \right).
\end{equation}

If $E_0$ is the energy largest in magnitude, then all the coefficients
of the excited states are fractions less (in magnitude) than 1.  
In that case, if we apply the Hamiltonian a large number of times, denoted by m, all terms excluding
the ground state term will vanish.
\begin{equation}
\mathcal{H}^m\lvert \psi \rangle =
		E_0^m \left(c_0 \lvert 0 \rangle + 
		c_1 \left(\frac{E_1}{E_0}\right)^m \lvert 1 \rangle +
		c_2\left(\frac{ E_2}{E_0}\right)^m \lvert 2 \rangle + \cdots \right)
		\approx E_0^m c_0 \lvert 0 \rangle
\end{equation}

{\color{red} Check the sign for this part.}
If the ground state energy is not the largest in magnitude, as is the case with the Heisenberg
model, we can manipulate the Hamiltonian slightly by adding or subtracting an
appropriately chosen constant term, $x$, in which case we will have
\begin{equation}
(x-\mathcal{H)}^m\lvert \psi \rangle =
		(E_0-x)^m \left(c_0 \lvert 0 \rangle + 
		c_1 \left(\frac{E_1-x}{E_0-x}\right)^m \lvert 1 \rangle  + \cdots \right)
		\approx (E_0-x)^m c_0 \lvert 0 \rangle.
\end{equation}

{\color{red} Comments to totally conclude up this section.}


%--------------------------------------------------------------------------------------------------------------------------
\section{The Hamiltonian and Bond Operators}

Throughout this thesis we will be looking at the Heisenberg model in one and two dimensions,
and so the Hamiltonian used will be the isotropic, antiferromagnetic Heisenberg 
Hamiltonian:
\begin{equation}
\mathcal{H}_{\rm{Heis}}=J\sum_{\langle i,j \rangle} \mathbf{S}_i\cdot \mathbf{S}_j
= J\sum_{\langle i,j \rangle}
	\left( S_i^z S_j^z + \tfrac{1}{2}\left[ S_i^+ S_j^- + S_i^- S_j^+ \right]\right),
\end{equation}
where the coupling constant $J$ is always positive, and $\sum_{\langle i,j \rangle}$ 
represents a sum over all nearest-neighbor pairs of sites.  
{\color{red} Make this better or get rid of it?}
If we apply this Hamiltonian to states in the $S^z$ basis, 
the first term will assign lower energy to pairs of sites with antiparallel spins,
while the remainder of the Hamiltonian will act to flip pairs of spins that are already 
antiparallel or annihilate states with parallel spins.

We slightly modify the Hamiltonian for use in the ground state projection scheme:

\begin{equation}
\mathcal{H}= \left(\mathcal{H}_{\rm{Heis}} - \tfrac{1}{4} \right)= \sum_{\langle i,j \rangle} 
	\left(\mathbf{S}_i\cdot \mathbf{S}_j - \tfrac{1}{4}\right)
	= - \sum_{\langle i,j \rangle} H_{ij}.
\end{equation}
Here the coupling constant $J$ is set to 1, and rewrite the Hamiltonian in terms of a list of
\it{bond operators}, \rm $H_{ij}$, where 
$H_{ij}=-\left(\mathbf{S}_i\cdot \mathbf{S}_j - \tfrac{1}{4}\right)$.

The effect of these bond operators acting on a valence bond basis state is 
surprisingly simple.  If a bond operator acts on two sites already joined by a valence
bond, it acts as the identity and does not changed the state.  If the two sites acted upon are 
not joined in a valence bond, the operator joins those two sites, and as a byproduct the 
two sites that were once joined to those sites form a valence bond themselves.
This is depicted in {\color{red} picture of VBs and bond ops acting on them and stuff.}
It can also be shown mathematically, but first let's examine the effect of bond operators
on a general spin 1/2 state.

We can rewrite the dot product of spin operators:
\begin{eqnarray}
\mathbf{S}_i\cdot \mathbf{S}_j = \tfrac{1}{2}\left[ \left(S_i + S_j\right)^2 -S_i^2-S_i^2 \right],
\end{eqnarray}
and since we are dealing with spin 1/2 particles, applying the $S^2$ operators to any state will
give 
\begin{equation}
S^2\lvert \psi\rangle = s(s+1)\lvert \psi \rangle = \tfrac{1}{2}(\tfrac{1}{2} + 1)\lvert \psi \rangle
	= \tfrac{3}{4}\lvert\psi\rangle
\end{equation}
for an arbitrary spin 1/2 state, $\lvert \psi\rangle$.  However, the $\left(S_i + S_j\right)^2$ 
operator has two different eigenvalues, or it could change the state
if the initial state is not one of the four total spin eigenstates.

Acting on an eigenstate of the total spin operator for two spins with a bond operator yields:
\begin{eqnarray}
H_{ij}\lvert \psi \rangle = \begin{cases}
	-\left(\tfrac{1}{2} \left[(0) - \tfrac{3}{4} - \tfrac{3}{4}\right] -\tfrac{1}{4}\right)\lvert\psi\rangle 
	= \lvert\psi\rangle \text{ for total spin 0}\\
	 -\left(\tfrac{1}{2} \left[(2) - \tfrac{3}{4}- \tfrac{3}{4}\right]-\tfrac{1}{4}\right)\lvert\psi\rangle
	 = 0 \text{ \;\;\;for total spin 1}
	 \end{cases}
	 \label{bop}
\end{eqnarray}
If we want to use the bond operators on valence bond basis states Eq.~(\ref{bop}) tells
us what happens when sites i and j are already joined in a valence bond, but we still
need to look at the case in which the sites are initially part of different valence bonds.
	


%--------------------------------------------------------------------------------------------------------------------------
\section{Single Projector}
\subsection{Measurements}
%--------------------------------------------------------------------------------------------------------------------------
\section{Double Projector}
%--------------------------------------------------------------------------------------------------------------------------
\section{Loop Moves}
%--------------------------------------------------------------------------------------------------------------------------
