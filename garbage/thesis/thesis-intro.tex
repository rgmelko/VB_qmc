% INTRODUCTION
%---------------------------
\chapter{Introduction}

%scaling\\
%universal quantities...\\
%The search for topological order\\
%looking at quantum critical points\\
%area law\\
%new attempts at entropies:\\
%- 
%\\

%According to G. Vidal: \\
%- the way tensor network states are constructed they can't have entropy scaling faster than area law\\

In this chapter we discuss different measures of entanglement and their scaling.
Then in Chapter 2 we describe quantum Monte Carlo algorithms in the valence bond basis.
Chapter 3 contains measurements of valence bond entanglement entropy as compared to standard von Neumann entanglement entropy.
In Chapter 4 we develop a method of measuring all Renyi entanglement entropies, excluding the von Neuman entanglement entropy, using valence bond quantum Monte Carlo.
In the final chapter we summarize the results of this research.

%---------------------------------------------------------------------------------------------------------
\section{Quantum Entanglement}
%---------------------------------------------------------------------------------------------------------
Entanglement is one important and interesting feature that differentiates quantum mechanical systems from classical systems.
In a pair of entangled spins/particles/photons/states the members of the system have information about the other member, even if they are spatially separated.
Two spins, for instance, are considered entangled if their joint state can not be written as the product of their individual states.  Of the following two states,
\begin{equation}
	\ket{\Psi_{1,2}} = \frac{1}{\sqrt{2}} \big( \ket{\up_1\up_2} + \ket{\dw_1\dw_2}\big) 
	\hspace{1cm} \text{and} \hspace{1cm}
	\ket{\Phi_{1,2}} = \frac{1}{\sqrt{2}} \big( \ket{\up_1\dw_2} - \ket{\dw_1\dw_2}\big), 
\end{equation}
only $\ket{\Phi_{1,2}}$ is separable, and can be rewritten as
\begin{equation}
	\ket{\Phi_{1,2}} = \frac{1}{\sqrt{2}}  \big( \ket{\up_1} - \ket{\dw_1}\big) \otimes \ket{\dw_2}.
\end{equation}
In the case of only two spins it is easy to see that states are separable (unentangled). As we add more spins to the system it becomes less apparent which  states are entangled. Furthermore, we can quantify the degree to which they are entangled beyond being either separable states, or non-separable states (entangled).

%---------------------------------------------------------------------------------------------------------
\section{Measures of Entanglement}
%---------------------------------------------------------------------------------------------------------

There are many different quantities that can be used to measure entanglement. Though not exhaustive a few measures include concurrence, logarithmic negativity, squashed entanglement, and entropy of entanglement.
All measures of entanglement share three properties\cite{Plenio2005}:
\begin{enumerate}
\item A bipartite entanglement measure $E(\rho)$ (measuring the entanglement between two qubits) is a mapping from density matrices to positive real numbers defined for states of arbitrary bipartite systems.
Usually a normalization factor is included such that the maximally entangled state,
\begin{equation}
\ket{\psi_{\rm max}}  = \frac{\ket{0,0} + \ket{1,1}}{\sqrt{2}},
\end{equation}
 of two qubits has $E(\ket{\psi_{\rm max}}\bra{\psi_{\rm max}}) = \ln(2)$.
 \item  $E(\rho)=0$ if the state $\rho$ is separable.
 \item $E(\rho)$ does not increase under local operators on either of the qubits.  That is, if we have an operator acting on only {\it one} of the qubit, the entanglement between the two qubits will never increase.
% \item For a pure state $\ket{\psi}\bra{\psi}$ the measure reduces to the von Neumann entanglement entropy.
\end{enumerate}
There is also a fourth condition, not satisfied by all measures of entanglement, requiring the measure to reduce to the von Neumann entanglement entropy for a system in a pure state.


%---------------------------------------------------------------------------------------------------------
\section{The von Neumann Entanglement Entropy}
%---------------------------------------------------------------------------------------------------------
For a system partitioned into two regions, A and B, the von Neumann entanglement entropy \vn is defined as
\begin{equation}
	\VN_\text{A} = -\rm{Tr}\left({\rho_A \ln \rho_A}\right),
\end{equation}
where $\rho_{\rm{A}}=\rm{Tr}_B\ket{\psi}\bra{\psi}$ is the reduced density matrix, the density matrix of the entire system with the degrees of freedom from region B traced out.
\vn is only suited to measure the entanglement of a pure state, as is the case for all the entanglement measures we will discuss in this thesis.
This is due to the way \vn measures entanglement; we begin with a pure state and trace out the degrees of freedom for some subregion B.  
We are left with $\rho_{\rm A}$ representing a, possibly mixed, state.
\vn can be rewritten in terms of the eigenvalues of the density matrix,
\begin{equation}
\VN_\text{A} = -\sum_i \left(\lambda_i \ln \lambda_i\right),
\end{equation}
which is quite similar to the form of thermodynamic entropy or Shannon information entropy. If $\rho_{\rm A}$ represents a pure state we {\bf know} what state of region A is in, and $\rho_{\rm A}$ will have one non-zero eigenvalue equal to 1 (so $\VN = 0$).
If $\rho_{\rm A}$ represents a mixed state, we will have some probability distribution of the possible states (so $\VN > 0$). The entropy of that distribution is used to quantify the entanglement between regions A and B.
In the maximally entangled case all eigenvalues are equal to $1/d$ (representing an equal probability of encountering each spin state), where $d$ is the dimension of the Hilbert space for region A ($2^N$ for a system of $N$ spin-1/2 particles). Thus the maximum entanglement entropy for a spin-$1/2$ system is $\VN= \ln (d) = N \ln(2)$. 
If the initial state starts off mixed instead of pure, the state would already have some entropy. The entanglement entropy would contain both the initially present classical uncertainty in the state as well as that due to tracing out region B.

The von Neumann entanglement entropy is part of a larger class of entanglement entropies called the generalized Renyi entanglement entropies.
The $n^\text{th}$ Renyi entanglement entropy $S_n$ is defined as
\begin{equation} \label{renyi}
 	S_n(\rho_{\rm A}) = \frac{1}{1-n}\ln\left[{\rm Tr}\left( \rho_{\rm A}^n \right) \right].
\end{equation}
%It may not be immediately apparent that $\VN = S_1$, but if we take the limit as $n\to 1$,
%\begin{equation}
%S_1(\rho_{\rm A}) = \lim_{n\to1}\frac{\ln\left[{\rm Tr}\left( \rho_{\rm A}^n \right) \right]}{1-n}
%	=\lim_{n\to1}\frac{\frac{d}{dn}\ln\left[{\rm Tr}\left( \rho_{\rm A}^n \right) \right]}{\frac{d}{dn}(1-n)}
%\end{equation}
Taking the limit $n\to1$ gives us $S_1 = \VN$.  These entanglement entropies have the property that for $n>m$, $S_n\ge S_m$; successive Renyi entropies give a lower bound on the previous entropies. They are useful measures to fully characterize the entanglement entropy of the system.

% \comment{cite some quantum information textbook}.


%---------------------------------------------------------------------------------------------------------
\section{Entanglement in Condensed Matter Systems}
%---------------------------------------------------------------------------------------------------------

Using measures of entanglement from quantum information to study interacting quantum many-body systems is a rapidly growing interdisciplinary topic \cite{Amico, intro}.
In systems that exhibit topological order, entanglement measures can be used to quantify ``hidden'' correlations \cite{wolf} and that order which is missed by standard measurements such as two-point correlation functions \cite{Bbob, KP, LW, Spectrum}.
Entanglement has the added benefit of being independent of the basis used to represent the condensed matter system.

%---------------------------------------------------------------------------------------------------------
\subsection{Scaling of Entanglement Entropy}
\label{1dcft}
%---------------------------------------------------------------------------------------------------------

The scaling of \vn is well understood in 1D systems \cite{ALreview} where, away from special critical points, entanglement between two regions scales as the size of the boundary between those regions.{\color{red}{Could expand this into a small blurb why. Talking about boundary.}} This ``area law'' \cite{Shredder} is expected to hold for the ground states of many interacting 2D quantum systems.

The scaling of entanglement entropy in 2D is also known to hold information about universal quantities.
In critical systems the area law scaling has universal additive logarithmic corrections \cite{Casini2007,Ryu}, and for topologically ordered systems there are universal additive constant corrections \cite{KP,LW}.

The 1D Heisenberg spin chain is critical, and the entanglement entropy is known to scale as \cite{Cardy} \cite{Zhou2006}
\begin{equation} \label{cft1d} 
	\VN_{\text{PBC}} = \frac{c}{3}\ln(x') + s_1
	\hspace{1cm}
	\VN_{\text{OBC}} = \frac{c}{6}\ln(2x') + \ln(g)+ s_1
\end{equation}
for systems with open and periodic boundary conditions, where $c$ is the central charge from conformal field theory, $g$ is a universal boundary term \cite{AffleckAndLudwig}, and $s_1$ is a model dependent constant. $x'$ is the conformal mapping for the number of sites included in region A, used to account for the finite size of the Heisenberg chains, when \eqref{cft1d} applies in the long chain-length limit.  $x \rightarrow x' = (L/\pi)\sin(\pi x/L)$ for PBC and $x \rightarrow 2x'$ for OBC, where L is the length of the chain.

For 2D Heisenberg spin systems with topological order, the entanglement entropy is known to scale as \cite{intro}
\begin{equation} \label{AL}
S_{\rm A} = \alpha \ell - n \gamma + \mathcal{O}(1/\ell),
\end{equation}
where $\ell$ is the length of the boundary, $\gamma$ is the topological entanglement entropy\cite{KP,LW}, and $n$ is related to the shape of the boundary.

\section{Measurement of Entanglement Entropy}

Many methods have been developed that are capable of measuring the entanglement entropy. In 1D DMRG and exact diagonalization are particularly powerful tools. They provide direct access to the density matrix and wavefunction. However exact diagonalization scales exponentially poorly as a function of system size and DMRG is primarily a tool useful for 1D and multleg ladders. To study and extract these universal quantities in 2D it is necessary to be able to measure entanglement entropy in a scalable type of simulation. New algorithms such as tensor-network states and projected entangled pair states, rely on states that are explicitly constructed to obey an area law \cite{PEPS1,PEPS2} but can scale to large system sizes. States scaling no faster than this area law can be accurately represented with matrix product states \cite{MPS_DMRG}. Quantum Monte Carlo is a method which scales well in any dimension and allows a choice of basis which is not constructed to obey any particular entropy scaling relationships. In this thesis we explore methods of measuring entanglement entropy in quantum Monte Carlo simulations in the valence bond basis.


