% INTRODUCTION
%---------------------------
\chapter{Introduction}

\comment{Find somewhere to cram in how we can't measure \vn with QMC because we don't have the wavefunction.}

%scaling\\
%universal quantities...\\
%The search for topological order\\
%looking at quantum critical points\\
%area law\\
%new attempts at entropies:\\
%- 
%\\

%According to G. Vidal: \\
%- the way tensor network states are constructed they can't have entropy scaling faster than area law\\

In this chapter discuss entanglement and the scaling of certain measures of entanglement.
Then in Chapter 2 we describe quantum Monte Carlo algorithms in the valence bond basis.
Chapter 3 contains measurements of a quantity called the valence bond entanglement entropy compared to the standard measure of entropy, the von Neumann entanglement entropy.
In Chapter 4 we develop a method of measuring the Renyi entanglement entropies, excluding the von Neuman entanglement entropy, using valence bond quantum Monte Carlo.
In the final chapter we summarize the results of this research.

%---------------------------------------------------------------------------------------------------------
\section{Entanglement}
%---------------------------------------------------------------------------------------------------------
\comment{What *is* entanglement?  What even is it?}
%---------------------------------------------------------------------------------------------------------
\section{Measures of Entanglement}
%---------------------------------------------------------------------------------------------------------
\comment{Talk about a few maybe?  Concurrence, Fidelity, Renyi Entanglement Entropies!}

\begin{equation} \label{renyi}
 	S_n(\rho_{\rm A}) = \frac{1}{1-n}\ln\left[{\rm Tr}\left( \rho_{\rm A}^n \right) \right]
\end{equation}


%---------------------------------------------------------------------------------------------------------
\section{The von Neumann Entanglement Entropy}
%---------------------------------------------------------------------------------------------------------

For a system divided into two regions, A and B, the von Neumann entanglement entropy is defined as
\begin{equation}
	\VN_\text{A} = -\rm{Tr}\left({\rho_A \ln \rho_A}\right),
\end{equation}
where $\rho_{\rm{A}}=\rm{Tr}_B\ket{\psi}\bra{\psi}$ is the reduced density matrix, that is, the density matrix of the entire system, with the degrees of freedom from region B traced out.
%---------------------------------------------------------------------------------------------------------
\section{Scaling of Entanglement Entropy}
%---------------------------------------------------------------------------------------------------------
In one dimensional interacting quantum systems, there are exact analytical results for the scaling of the von Neumann entanglement entropy know from Conformal field theory (CFT).
These results show that away from critical points $\VN_\text{A}$ scales like the size of the boundary between regions A and B.  This scaling of entanglement entropy is called the {\it area law} \cite{Shredder}, and it is expected to hold for the ground states of two dimensional unfrustrated spin systems \cite{DeBeaudrap2010} \comment{another ref?}.

\comment{Should I put the area law stuff *in* the CFT section?  That's what happens in the paper.  It says that CFT results give the area law for 1D and it's believed to hold in 2D too.}

\comment{Don't forget about corrections to the area law.}

\comment{Also, consequences of things not following the area law.  
(harder to simulate if entanglement scales faster than the area law).}

%---------------------------------------------------------------------------------------------------------
\subsection{1D Critical Systems}
\label{1dcft}

\cite{Cardy} \cite{Zhou2006}

described by CFT in the limit of the length of the chain, $L \rightarrow \infty$.
\comment{Something about the "conformal mapping" $x \rightarrow x' = (L/\pi)\sin(\pi x/L)$ for PBC
and $x \rightarrow 2x'$ for OBC.}
\begin{equation} \label{cft1d}
	\VN_{\text{PBC}} = \frac{c}{3}\ln(x') + S_1
	\hspace{1cm}
	\VN_{\text{OBC}} = \frac{c}{6}\ln(2x') + \ln(g)+ S_1
\end{equation}
\comment{Change $S_1$ to some other variable name!!!}
\change{where $c$ is the central charge of the CFT, $S_1$ is a model-dependent constant, and $g$ is Affleck and Ludwig's universal boundary term \cite{AffleckAndLudwig}.} 
\comment{$\leftarrow$ the {\bf most} plagiarized? }

\comment{Talk about what values $c$ can take.  From Cardy and Zhou it seems like $c=1/2,1$, but in our case it's 1.} 




