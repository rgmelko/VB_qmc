% UW Thesis Template for LaTeX 
% Last Updated July 8, 2009 by Stephen Carr, IST Client Services
% FOR ASSISTANCE, please send mail to rt-IST-CSmathsci@ist.uwaterloo.ca

% Effective October 2006, the University of Waterloo 
% requires electronic thesis submission. See the UW thesis regulations at
% http://www.grad.uwaterloo.ca/Thesis_Regs/thesistofc.asp.
% However, many faculties/departments also require one or more printed
% copies. This template attempts to satisfy requirements for both types of output. 
% It is based on the standard "report" document class. The "book" document 
% class can be substituted if you have a very large, multi-part thesis.

% DISCLAIMER
% To the best of our knowledge, this template satisfies the current UW requirements.
% However, it is your responsibility to assure that you have met all 
% requirements of the University and your particular department.
% Many thanks to Colin Alie for assistance in preparing this updated template.

% -----------------------------------------------------------------------

% By default, output is produced that is geared toward generating a PDF 
% version optimized for viewing on an electronic display, including 
% hyperlinks within the PDF.
 
% E.g. to process a thesis called "mythesis.tex" based on this template, run:

% pdflatex mythesis	-- first pass of the pdflatex processor
% bibtex mythesis	-- generates bibliography from .bib data file(s) 
% pdflatex mythesis	-- fixes cross-references, bibliographic references, etc
% pdflatex mythesis	-- fixes cross-references, bibliographic references, etc

% If you use the recommended LaTeX editor, Texmaker, you would open the mythesis.tex
% file, then click the pdflatex button. Then run BibTeX (under the Tools menu).
% Then click the pdflatex button two more times. If you have an index as well,
% you'll need to run MakeIndex from the Tools menu as well, before running pdflatex
% the last two times.

% N.B. The "pdftex" program allows graphics in the following formats to be
% included with the "\includegraphics" command: PNG, PDF, JPEG, TIFF
% Tip 1: Generate your figures and photos in the size you want them to appear
% in your thesis, rather than scaling them with \includegraphics options.
% Tip 2: Any drawings you do should be in scalable vector graphic formats:
% SVG, PNG, WMF, EPS and then converted to PNG or PDF, so they are scalable in
% the final PDF as well.
% Tip 3: Photographs should be cropped and compressed so as not to be too large.

% To create a PDF output that is optimized for double-sided printing: 
%
% 1) comment-out the \documentclass statement in the preamble below, and
% un-comment the second \documentclass line.
%
% 2) change the value assigned below to the boolean variable
% "ElectronicVersion" from "true" to "false".

% --------------------- Start of Document Preamble -----------------------

% Specify the document class, default style attributes, and page dimensions
% For hyperlinked PDF, suitable for viewing on a computer, use this:
\documentclass[letterpaper,12pt,titlepage,oneside,final]{report}
 
% For un-hyperlinked PDF, suitable for double-sided printing, use this:
%\documentclass[letterpaper,12pt,titlepage,openright,twoside,final]{report}

% Some LaTeX commands I define for my own nomenclature.
% If you have to, it's better to change nomenclature once here than in a 
% million places throughout your thesis!
\newcommand{\package}[1]{\textbf{#1}} % package names in bold text
\newcommand{\cmmd}[1]{\textbackslash\texttt{#1}} % command name in tt font 
\newcommand{\href}[1]{#1} % does nothing, but defines the command so the
    % print-optimized version will ignore \href tags (redefined by hyperref pkg).
%\newcommand{\texorpdfstring}[2]{#1} % does nothing, but defines the command
% Anything defined here may be redefined by packages added below...

% This package allows if-then-else control structures.
\usepackage{ifthen}
\newboolean{ElectronicVersion}
\setboolean{ElectronicVersion}{true} % CHANGE THIS VALUE AS REQUIRED

%\usepackage{nomencl} % For a nomenclature (optional; available from ctan.org)
\usepackage{amsmath,amssymb,amstext} % Lots of math symbols and environments
\usepackage[pdftex]{graphicx} % For including graphics N.B. pdftex graphics driver 
\usepackage[bindingoffset=0.125in,centering,includeheadfoot,margin=1in]{geometry}
% Hyperlinks make it very easy to navigate an electronic document.
% The following statement places "bookmarks" in the PDF document
% linking Table of Contents, List of Figures, List of Tables, and
% List of References entries to their appearance in the body of the
% text.  In addition, this is where one would specify the thesis title
% and author as they appear in the properties of the PDF document.

% Use the "hyperref" package only for the electronic version:
% N.B. HYPERREF MUST BE THE LAST PACKAGE LOADED; ADD ADDITIONAL PKGS ABOVE
\ifthenelse{\boolean{ElectronicVersion}}{
    \usepackage[letterpaper=true,pdftex,bookmarks,pagebackref,
	plainpages=false, % Needed if Roman numbers in frontpages
        pdfpagelabels=true, % Adds page number as label in Acrobat's page count 
        pdftitle=UW\ LaTeX\ Thesis\ Template, % CHANGE THIS TEXT!
        pdfauthor=Pat\ Neugraad	 % CHANGE THIS TEXT!
        ]{hyperref}}{}

% Setting up the page margins...
% UW thesis requirements specify a minimum of 1 inch (72pt) margin at the
% top, bottom, and outside page edges and a 1.125 inch (81pt) gutter
% margin (on binding side). While this is not an issue for electronic
% viewing, a PDF may be printed, and so we have the same page layout for
% both printed and electronic versions, we leave the gutter margin in.
% The standard method of setting margins is with the \setlength command.
% We have used the "geometry" package, added above, to do it in one line.
% For completeness, the appropriate \setlength commands are show here, 
% commented out.
% Set margins to minimum permitted by UW thesis regulations:
%\setlength{\marginparwidth}{0pt} % width of margin notes
% N.B. If margin notes are used, you must adjust \textwidth, \marginparwidth
% and \marginparsep so that the space left between the margin notes and page
% edge is less than 15 mm (0.6 in.)
%\setlength{\marginparsep}{0pt} % width of space between body text and margin notes
%\setlength{\evensidemargin}{0.5in} % Adds 1/8 in. to binding side of all 
% even-numbered pages when the "twoside" printing option is selected
%\setlength{\oddsidemargin}{0.5in} % Adds 1/8 in. to the left of all pages
% when "oneside" printing is selected, and to the left of all odd-numbered
% pages when "twoside" printing is selected
%\setlength{\textwidth}{6.0in} % assuming US letter paper (8.5 in. x 11 in.) and 
% side margins as above
\raggedbottom

% The following statement specifies the amount of space between
% paragraphs. Other reasonable specifications are \bigskipamount and \smallskipamount.
\setlength{\parskip}{\medskipamount}

% The following statement controls the line spacing.  The default
% spacing corresponds to good typographic conventions and only slight
% changes (e.g., perhaps "1.2"), if any, should be made.
\renewcommand{\baselinestretch}{1} % this is the default line space setting

% By default, each chapter will start on a recto (right-hand side)
% page. We also force each section of the front pages to start on 
% a recto page by inserting \cleardoublepage commands in the 
% uw-ethesis-frontpgs.tex file, included below, with the \input command.
% In many cases, this will require that the verso page be
% blank and, while it should be counted, a page number should not be
% printed.  The following statements ensure a page number is not
% printed on an otherwise blank verso page.
\let\origdoublepage\cleardoublepage
\newcommand{\clearemptydoublepage}{%
  \clearpage{\pagestyle{empty}\origdoublepage}}
\let\cleardoublepage\clearemptydoublepage

%======================================================================
%   L O G I C A L    D O C U M E N T -- the content of your thesis
%======================================================================
\begin{document}

% For a large document, it is a good idea to divide your thesis
% into several files, each one containing one chapter.
% To illustrate this idea, the "front pages" (i.e., title page,
% declaration, borrowers' page, abstract, acknowledgements,
% dedication, table of contents, list of tables, list of figures,
% nomenclature) are contained within the file "uw-ethesis-frontpgs.tex" which is
% included into the document by the following statement.
%----------------------------------------------------------------------
% FRONT MATERIAL
%----------------------------------------------------------------------
% T I T L E   P A G E
% -------------------
% This file goes along with the master LaTeX file uw-ethesis.tex
% Last updated May 27, 2009 by Stephen Carr, IST Client Services
% The title page is counted as page `i' but we need to suppress the
% page number.  We also don't want any headers or footers.
\pagestyle{empty}
\pagenumbering{roman}

% The contents of the title page are specified in the "titlepage"
% environment.
\begin{titlepage}
        \begin{center}
        \vspace*{1.0cm}

        \Huge
        {\bf UW E-Thesis Template with Examples}

        \vspace*{1.0cm}

        \normalsize
        by \\

        \vspace*{1.0cm}

        \Large
        Pat Neugraad \\

        \vspace*{3.0cm}

        \normalsize
        A thesis \\
        presented to the University of Waterloo \\ 
        in fulfillment of the \\
        thesis requirement for the degree of \\
        Master of Science \\
        in \\
        Zoology \\

        \vspace*{2.0cm}

        Waterloo, Ontario, Canada, 2007 \\

        \vspace*{1.0cm}

        \copyright\ Pat Neugraad 2007 \\
        \end{center}
\end{titlepage}

% The rest of the front pages should contain no headers and be numbered using Roman numerals starting with `ii'
\pagestyle{plain}
\setcounter{page}{2}

\cleardoublepage % Ends the current page and causes all figures and tables that have so far appeared in the input to be printed.
% In a two-sided printing style, it also makes the next page a right-hand (odd-numbered) page, producing a blank page if necessary.
 


% D E C L A R A T I O N   P A G E
% -------------------------------
  % The following is the sample Delaration Page as provided by the GSO
  % December 13th, 2006.  It is designed for an electronic thesis.
  \noindent
I hereby declare that I am the sole author of this thesis. This is a true copy of the thesis, including any required final revisions, as accepted by my examiners.

  \bigskip
  
  \noindent
I understand that my thesis may be made electronically available to the public.

\cleardoublepage
%\newpage

% A B S T R A C T
% ---------------

\begin{center}\textbf{Abstract}\end{center}

This is the abstract.

\cleardoublepage
%\newpage

% A C K N O W L E D G E M E N T S
% -------------------------------

\begin{center}\textbf{Acknowledgements}\end{center}

I would like to thank all the little people who made this possible.
\cleardoublepage
%\newpage

% D E D I C A T I O N
% -------------------

\begin{center}\textbf{Dedication}\end{center}

This is dedicated to the one I love.
\cleardoublepage
%\newpage

% T A B L E   O F   C O N T E N T S
% ---------------------------------
\tableofcontents
\cleardoublepage
%\newpage

% L I S T   O F   T A B L E S
% ---------------------------
\listoftables
\addcontentsline{toc}{chapter}{List of Tables}
\cleardoublepage
%\newpage

% L I S T   O F   F I G U R E S
% -----------------------------
\listoffigures
\addcontentsline{toc}{chapter}{List of Figures}
\cleardoublepage
%\newpage

% L I S T   O F   S Y M B O L S
% -----------------------------
% \renewcommand{\nomname}{Nomenclature}
% \addcontentsline{toc}{chapter}{\textbf{Nomenclature}}
% \printglossary
%\cleardoublepage
% \newpage

% Change page numbering back to Arabic numerals
\pagenumbering{arabic}

 

%----------------------------------------------------------------------
% MAIN BODY
%----------------------------------------------------------------------
% Because this is a short document, and to reduce the number of files
% needed for this template, the chapters are not separate
% documents as suggested above, but you get the idea. If they were
% separate documents, they would each start with the \chapter command, i.e, 
% do not contain \documentclass or \begin{document} and \end{document} commands.
%======================================================================
\chapter{Introduction}
%======================================================================
In the beginning there was $\pi$:

\begin{equation}
   e^{\pi i} + 1 = 0  \label{eqn_pi}
\end{equation}

%----------------------------------------------------------------------
\section{State of the Art}
%----------------------------------------------------------------------

See equation \ref{eqn_pi} on page \pageref{eqn_pi}.\footnote{A famous equation.}

%======================================================================
\chapter{Observations}
%======================================================================

This would be a good place for some figures and tables.

Some notes on figures and photographs\ldots

\begin{itemize}
\item A well-prepared PDF should be 
  \begin{enumerate}
    \item Of reasonable size, {\it i.e.} photos cropped and compressed.
    \item Scalable, to allow enlargment of text and drawings. 
  \end{enumerate} 
\item Photos must be bit maps, and so are not scaleable by definition. TIFF and
BMP are uncompressed formats, while JPEG is compressed. Most photos can be
compressed without losing their illustrative value.
\item Drawings that you make should be scalable vector graphics, \emph{not} 
bit maps. Some scalable vector file formats are: EPS, SVG, PNG, WMF. These can
all be converted into PNG or PDF, that pdflatex recognizes. Your drawing 
package probably can export to one of these formats directly. Otherwise, a 
common procedure is to print-to-file through a Postscript printer driver to 
create a PS file, then convert that to EPS (encapsulated PS, which has a 
bounding box to describe its exact size rather than a whole page). 
Programs such as GSView (a Ghostscript GUI) can create both EPS and PDF from PS files.
Appendix~\ref{AppendixA} shows how to generate properly sized Matlab plots and save them as PDF.
\item It's important to crop your photos and draw your figures to the size that
you want to appear in your thesis. Scaling photos with the 
includegraphics command will cause loss of resolution. And scaling down 
drawings may cause any text annotations to become too small.
\end{itemize}
 
For more information on \LaTeX\, see the UW Skills for the Academic Workplace 
course notes at \href{http://saw.uwaterloo.ca/latex}{saw.uwaterloo.ca/latex}. 
\footnote{
Note that while it is possible to include hyperlinks to external documents,
it is not wise to do so, since anything you can't control may change over time. 
It \emph{would} be appropriate and necessary to provide external links to 
additional resources for a multimedia ``enhanced'' thesis. 
But also note that if the \package{hyperref} package is not included, 
as for the print-optimized option in this thesis template, any \cmmd{href} 
commands in your logical document are no longer defined.
A work-around employed by this thesis template is to define a dummy \cmmd{href} 
command (which does nothing) in the preamble of the document, 
before the \package{hyperref} package is included. 
The dummy definition is then redifined by the
\package{hyperref} package when it is included.
}

The classic book by Leslie Lamport \cite{lamport.book}, author of \LaTeX , is worth a look too, and the many available add-on packages are described by 
Goossens \textit{et al} \cite{goossens.book}. Some on-line documentation is linked
to from \href{http://saw.uwaterloo.ca/latex}{saw.uwaterloo.ca/latex}.

% The \appendix statement indicates the beginning of the appendices.
\appendix
% Add a title page before the appendices and a line in the Table of Contents
\chapter*{APPENDICES}
\addcontentsline{toc}{chapter}{APPENDICES}
%======================================================================
\chapter[PDF Plots From Matlab]{Matlab Code for Making a PDF Plot}
\label{AppendixA}
% Tip 4: Example of how to get a shorter chapter title for the Table of Contents 
%======================================================================
\section{Using the GUI}
Properties of Matab plots can be adjusted from the plot window via a graphical interface. Under the Desktop menu in the Figure window, select the Property Editor. You may also want to check the Plot Browser and Figure Palette for more tools. To adjust properties of the axes, look under the Edit menu and select Axes Properties.

To set the figure size and to save as PDF or other file formats, click the Export Setup button in the figure Property Editor.

\section{From the Command Line} 
All figure properties can also be manipulated from the command line. Here's an example: 
\begin{verbatim}
x=[0:0.1:pi];
hold on % Plot multiple traces on one figure
plot(x,sin(x))
plot(x,cos(x),'--r')
plot(x,tan(x),'.-g')
title('Some Trig Functions Over 0 to \pi') % Note LaTeX markup!
legend('{\it sin}(x)','{\it cos}(x)','{\it tan}(x)')
hold off
set(gca,'Ylim',[-3 3]) % Adjust Y limits of "current axes"
set(gcf,'Units','inches') % Set figure size units of "current figure"
set(gcf,'Position',[0,0,6,4]) % Set figure width (6 in.) and height (4 in.)
cd n:\thesis\plots % Select where to save
print -dpdf plot.pdf % Save as PDF
\end{verbatim}

%----------------------------------------------------------------------
% END MATERIAL
%----------------------------------------------------------------------

% B I B L I O G R A P H Y
% -----------------------

% The following statement selects the style to use for references.  It controls the sort order of the entries in the bibliography and also the formatting for the in-text labels.
\bibliographystyle{plain}
% This specifies the location of the file containing the bibliographic information.  
% It assumes you're using BibTeX (if not, why not?).
\bibliography{template}
% The following statement causes the title "References" to be used for the biliography section:
\renewcommand{\bibname}{References}
\addcontentsline{toc}{chapter}{\textbf{References}}
% Tip 5: You can create multiple .bib files to organize your references. 
% Just list them all in the \bibliogaphy command, separated by commas (no spaces).

% The following statement causes the specified references to be added to the bibliography% even if they were not 
% cited in the text. The asterisk is a wildcard that causes all entries in the bibliographic database to be included (optional).
\nocite{*}

\end{document}
